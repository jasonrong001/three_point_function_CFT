\documentclass[a4paper,11pt]{article}
\pdfoutput=1 % if your are submitting a pdflatex (i.e.~if you have
% images in pdf, png or jpg format)

\usepackage{jheppub,amsthm} % for details on the use of the package, please
% see the JHEP-author-manual

\usepackage[utf8]{inputenc}

\usepackage{booktabs,multirow, quotes}

\usepackage[english]{babel}
\usepackage{amsmath,amssymb,graphicx,xcolor,alltt}
\usepackage{framed}

%%%%%%%%%% Start TeXmacs macros
%\catcode`\<=\active \def<{
%\fontencoding{T1}\selectfont\symbol{60}\fontencoding{\encodingdefault}}
\newcommand{\longminus}{{-\!\!-}}
\newcommand{\tmop}[1]{\ensuremath{\operatorname{#1}}}


\newcommand{\beq}{\begin{equation}}
\newcommand{\eeq}{\end{equation}}
\newcommand{\bea}{\begin{eqnarray}}
\newcommand{\ea}{\end{eqnarray}}


\theoremstyle{remark}
\newtheorem{remark}{Remark}[section]
\newtheorem{example}{Example}[section]
\theoremstyle{theorem}
\newtheorem{lemma}{Lemma}[section]

%%%%%%%%%% End TeXmacs macros
\def\red{\color{red}}

\newcommand{\SR}[1]{{\red \bf [{SR:} #1]}}
\newcommand{\junchen}[1]{\textcolor{blue}{\textbf{[JR: #1]}}}

\makeatletter
\def\@fpheader{\ }
\makeatother

\title{Conformal symmetry test with Rydberg atoms - feasibility study}

 
\author{Junchen Rong$^1$,}
\author{Slava Rychkov$^2$}

% The "\note" macro will give a warning: "Ignoring empty anchor..."
% you can safely ignore it.

\affiliation{$^1$ CPHT, CNRS, \'Ecole Polytechnique, Institut Polytechnique de Paris, Palaiseau, France\\
$^2$IHES, Bures-sur-Yvette}


% e-mail addresses: one for each author, in the same order as the authors
\emailAdd{junchen.rong@polytechnique.edu}
\emailAdd{slava@ihes.fr}




\abstract{This is a note on the measurement of 3pt function using Rydberg atom experiments.}



\begin{document} 
	
	\maketitle
	
	\flushbottom


\flushbottom


\section{Introduction}
Conformal Field Theory (CFT) defines predicts correlation functions of local operators. We will be dealing with a particular 1+1D CFT called the Ising CFT. This CFT famously has three primary operators: $1$ (identity), $\sigma$, and $\epsilon$, of scaling dimensions $0$, $1/8$, $1$. The equal-time correlation functions of $\sigma$ and $\epsilon$ are predicted to be~\cite{polyakov1969microscopic}:
\begin{align}
	\langle\sigma(x_1)\sigma(x_2)\rangle &=\frac{1}{|x-y|^{2\Delta_{\sigma}}}, \label{eq:ss}\\
\langle\epsilon(x_1)\epsilon(x_2)\rangle &=\frac{1}{|x-y|^{2\Delta_{\epsilon}}},\label{eq:ee} \\
	\langle\sigma(x_1)\sigma(x_2)\epsilon(x_3)\rangle &=\frac{C_{\sigma\sigma\epsilon}}{|x_1-x_2|^{2\Delta_{\sigma}-\Delta_{\epsilon}}|x_1-x_3|^{\Delta_{\epsilon}}|x_2-x_3|^{\Delta_{\epsilon}}},\quad C_{\sigma\sigma\epsilon}=1/2. \label{eq:ess}
\end{align}
These equations are written for equal-time correlation functions on an infinite-length straight line, Fig.~\ref{fig:straight-line-circle}(a). On a circle of length $L$, Fig.~\ref{fig:straight-line-circle}(b), we should use the same formulas with the substitution
\beq
|x_i-x_j|\to |x_i-x_j|_L ,\quad |x|_L:= \frac{L}{\pi} \sin\left(\frac{\pi}{L} |x|\right)
\label{eq:xL}
\eeq
\begin{figure}[h]
	\centering
	\includegraphics[width=100pt]{fig-straight-line-circle.jpeg}
	\caption{Equal-time correlations on infinite-length straight line (a) and circle of length $L$ (b)}
	\label{fig:straight-line-circle}
\end{figure}


In Section \ref{sec:model} we describe a microscopic model realizable with Rydberg atoms, which reduces at its quantum critical point to the Ising model CFT. 

%In Section \ref{sec:local-operators}:

\section{Microscopic model}
\label{sec:model}
Consider a chain of $L$ Rydberg atoms equally spaced along a circle:
\beq
	\includegraphics[width=100pt]{fig-circle.jpeg}
	\label{fig:circle}
\eeq
We describe the states in the 0-1 basis, where $|0\rangle$ ($|1\rangle$) on site $i$ corresponds to the $i$-th atom to be in the ground/Rydberg excited state.
The creation operator $a_i^\dagger$ maps $|0_i\rangle$ to $|1_i\rangle$. We have the occupation number of site $i$.
We refer to $|0\rangle$ as empty and $|1\rangle$ as occupied site. 

Using the same conventions as in~\cite{FangFang:2024eaw},  we write a realistic Hamiltonian where the atoms in the excited states interact with a repulsive $1/r^6$ interactions:
\begin{equation}
    \hat{H}=\frac{\Omega}{2} \sum_i (a_i^{\dagger}+a_i) - \Delta \sum_i n_i + C_6\sum_{ij}\frac{1}{(d_{ij})^6}  n_{i}n_{j}.
    \label{eq:model}
\end{equation}
Here $d_{ij}$ is the distance between the $i$-th and $j$-th sites,
 \beq   d_{ij}=a \frac{\sin(\frac{|i-j|}{L}\pi)}{\sin(\frac{1}{L}\pi)},
\eeq
where $a$ is the lattice constant. We will set $a=1$ for later discussion. For the effects of distance variation see Sec.~\ref{sec:distvar}.

The sign of $\Omega$ is not important, as it can flipped by redefining $|1\rangle \to -|1\rangle$.
%\begin{align}
%d^{\dagger}|0\rangle \rightarrow -d^{\dagger}|0\rangle. 
%\end{align}
The experimentally accessible parameter space is, roughly,
\begin{align}
\Omega/C_6 \in [0,5],\quad \Delta/C_6 \in[-5,5].
\end{align}
The phase diagram near the Ising transition is reported in Fig.~\ref{phasediagramHBlong}.
\begin{figure}[htbp]
\centering
\includegraphics[scale=0.5]{phase_diagram_SSF_LR.png}
\caption{The phase diagram of the Rydberg model \eqref{eq:model}. The red circle corresponds to the point studied in~\cite{FangFang:2024eaw}.}
\label{phasediagramHBlong}
\end{figure}

It is useful to contrast this model with the transverse field Ising model (TFIM). TFIM has lattice onsite $\mathbb{Z}_2$ symmetry. This corresponds to the $\mathbb{Z}_2$ symmetry of the CFT under which $\sigma\to-\sigma$ while $\epsilon$ is invariant. In contrast, the Rydberg atom model \eqref{eq:model} does not have onsite $\mathbb{Z}_2$ symmetry. Instead, $\mathbb{Z}_2$ symmetry of the CFT will emerge from a lattice translation by one lattice unit. (In the broken phase, the expectation value $\langle n_i \rangle$ is periodic with period 2, and shift by one lattice unit permutes the two ground states.) This will have to be taken into account when constructing lattice operators which represent CFT operators.

\section{Correlation functions: comparing CFT and DMRG}

\subsection{Two-point function of $\sigma$}

Operator $\sigma$ has low scaling dimension. Its two-point function decreases with the distance, but only quite slowly. This is the easiest nontrivial observable to measure experimentally. We also believe that this will be a great way to tune the system to the critical point (see Sec.~\ref{sec:tuning}).

The lattice operator that represents $\sigma$ is taken to be
\beq
\sigma(i+1/2) = c_1 (-1)^i (n_i -n_{i+1})
\label{eq:sigmalat}
\eeq
This equation is a good chance to understand the logic of matching local lattice operators and local CFT operators. Any local lattice operators corresponds to a linear combination of CFT operators and their derivatives. At large distances derivatives contribute less to correlation functions than the operators without derivatives, but still it's good to suppress them to get cleaner results. The linear combination $n_i -n_{i+1}$ in \eqref{eq:sigmalat} kills some unwanted terms, enhancing the contribution of $\sigma$. In the l.h.s. of \eqref{eq:sigmalat}, we put the CFT operator $\sigma$ at point $x=i+1/2$, which is the middle of the block of two lattice points $i,i+1$.\footnote{\label{note:imperfect} Note that it's still not an exact CFT $\sigma$, but a much better approximation than, say, $(-1)^in_i$. This discussion can be made much more detailed and theoretical; here we focus on the essential details.} When lattice operators are shifted by one unit, this gives a $\mathbb{Z}_2$ transformation in CFT, which would map $\sigma\to-\sigma$. To compensate it we include $(-1)^i$ factor in the r.h.s. of \eqref{eq:sigmalat}. Finally, CFT operators conventionally have unit normalization, see coefficient 1 in \eqref{eq:ss}. Lattice operators do not come with any natural normalization. We include constant $c_1$ in \eqref{eq:sigmalat} to achieve this. This constant cannot be determined from first principles; it will need to be determined by a fit.

\begin{figure}[htbp]
	\centering
	\includegraphics[scale=0.4]{fig-sigma-sigma.pdf}
	\includegraphics[scale=0.4]{fig-sigma-sigma-error.pdf}
	\caption{Left: The two-point function $\langle \sigma(0)\sigma(x)\rangle$ in the range $2\le |x|\le 20$, plotted in log-log-scale as a function of $|x|_L$, see Eq.~\eqref{eq:xL}. Solid blue line - exact CFT prediction $(|x|_L)^{-0.25}$. Red dots - our determination using DMRG and lattice operator \eqref{eq:sigmalat}. We only show the DMRG result at separation $|x|\ge 2$, where the lattice operators don't overlap. Right: Relative error of DMRG vs CFT, i.e. $|f_{\rm DMRG}/f_{\rm CFT}-1|$.}
	\label{fig:sigma-sigma}
\end{figure}

In Fig.~\ref{fig:sigma-sigma} we show the comparison between the exact CFT two-point function $\langle \sigma\sigma\rangle$ and the two-point function of lattice operators, as a function of the distance. We used DMRG to find the ground state of the $L=40$ chain, with Hamiltonian parameters:
\beq
\Omega/C_6=0.18, \quad \Delta/C_6=0.1843
\eeq
We then used lattice operator \eqref{eq:sigmalat} to measure the two-point function. The constant $c_1=1.32854$. 

In Fig.~\ref{fig:sigma-sigma}, we see that the two-point function determined in a numerical simulation differs by 2.5\% from the CFT prediction at the shortest distance $|x|=2$ where the lattice operators don't overlap, but the deviation goes down quickly at larger distances. Theoretically, the small discrepancy which goes to zero as $x\to\infty$ is due to two effects: 1) the microscopic model only becomes CFT at large distances; 2) the operator \eqref{eq:sigmalat} we constructed is imperfect, note \ref{note:imperfect}.

\subsection{Two-point function of $\epsilon$}


Similarly, we can map the CFT operator $\epsilon$ to the lattice operator. In this case it's a bit harder to construct the needed lattice operator, and some naive choices give poor results (not shown). Here is one operator which 1) works reasonably well in simulations and 2) is suitable for experimental work, as we will explain below:
\beq
\epsilon(i+1/2)=c_2 \big[ (n_{i-1} -n_{i})(n_{i+1} -n_{i+2}) - b \big], \quad b =\langle (n_{i-1} -n_{i})(n_{i+1} -n_{i+2}) \rangle\,.
\label{eq:epslat}
\eeq
To understand the expression of this operator note that it's obtained by putting next to each other two lattice operators for $\sigma$ from \eqref{eq:sigmalat}. Naively, you can think of $\epsilon$ as a "renormalized square" of the CFT operator $\sigma$, and \eqref{eq:epslat} corresponds to this intuition. In \eqref{eq:epslat}, we subtract the constant $b$ which corresponds to subtracting the identity contribution. This is needed because the CFT operators apart from the identity must have zero one point functions.
In \eqref{eq:sigmalat} we did not need to subtract it because it was automatically zero. The meaning of the constant $c_2$ is the same as $c_1$ in \eqref{eq:sigmalat}.

The DMRG result are summarised in Fig.~\ref{fig:eps-eps}. This is for the same Hamiltonian parameters as in \eqref{fig:sigma-sigma}. The constants are $b =0.488577$, $c_2 = 2.05816$. Because operator \eqref{eq:epslat} is constructed from 4 points, we only show results for $|x|\ge 4$ where the lattice operators don't overlap. Comparing this plot with the one for $\sigma$, we see that the two-point function of $\epsilon$ decreases with the distance much faster, due to larger scaling dimension of $\epsilon$. We may expect that the experimental measurement of this two-point function will be more challenging. This will be discussed below. The relative error of DMRG vs CFT is also larger, reaching 30\% at minimal non-overlapping separation, although also going quickly to zero as $x$ in increased.

\begin{figure}[htbp]
	\centering
	\includegraphics[scale=0.4]{fig-eps-eps.pdf}
		\includegraphics[scale=0.4]{fig-eps-eps-error.pdf}
	\caption{Left: The two-point function $\langle \epsilon(0)\epsilon(x)\rangle$ for $4\le |x|\le 20$, plotted in log-log-scale as a function of $|x|_L$, see Eq.~\eqref{eq:xL}. Solid blue line - exact CFT prediction $(|x|_L)^{-2}$. Red dots - our determination using DMRG and lattice operator \eqref{eq:epslat}. We only show the DMRG result at separation $|x|\ge 4$, because for smaller $|x|$ the lattice operators overlap. Right: Relative error of DMRG vs CFT.}
	\label{fig:eps-eps}
\end{figure}

\subsection{Three-point function $\langle \epsilon \sigma\sigma\rangle$}

Let us now discuss the three point function $\langle\epsilon(x_1) \sigma(x_2)\sigma(x_3)\rangle $.  We can set $x_1=0$ by translation invariance. The plots for the CFT prediction on the circle of length $L=40$ is shown in Fig.~\ref{fig:eps-sig-sig-theory}, as a 3D plot and as a contour plot . It is a function which blows up as $x_2$ or $x_3$ approach 0, and goes to zero as $x_2\to x_3$, see Eq.~\eqref{eq:ess}. This function is symmetric under the interchange of $x_2$ as $x_3$ and also under the flip of the circle orientation which maps $x_2\to L-x_2$, $x_3\to L-x_3$.

\begin{figure}[htbp]
	\centering
	\includegraphics[width=0.4\textwidth]{fig-eps-sig-sig-theory-3D.pdf}
	\includegraphics[width=0.4\textwidth]{fig-eps-sig-sig-theory.pdf}
	\caption{Left: The three-point function $\langle \epsilon(0)\sigma(x_2)\sigma(x_3)\rangle$ on a circle of length $L=40$. Contour plot  of the same function.} 
	\label{fig:eps-sig-sig-theory}
\end{figure}

In Fig.~\ref{fig:eps-sig-sig} we show the computation of the same correlation function using DMRG. We also show the relative error of DMRG w.r.t.~CFT. In DMRG, we only show the result for the ranges $|x_2|,|x_3|\ge 3$, $|x_2-x_3|\ge 2$ where the lattice operators don't overlap. We see that the agreement between CFT and DMRG is better than 10\% throughout a large region, shown with a red dashed line in the left plot. Note that the three-point function itself varies by factor 10 within the same region. In this test we used exactly the same Hamiltonian parameters and the same lattice operators for $\sigma$ and $\epsilon$ as in previous plots, i.e.~no further tuning is necessary. Thus we agree not only the shape of the CFT prediction, but also with the overall normalization (i.e.~the coefficient $C_{\sigma\sigma\epsilon}$ in Eq.~\eqref{eq:ess}.) 

\begin{figure}[htbp]
	\centering
		\includegraphics[width=0.4\textwidth]{fig-eps-sig-sig-DMRG.pdf}
		\includegraphics[width=0.4\textwidth]{fig-eps-sig-sig-relative-error.pdf}
	\caption{Left: The three-point function $\langle \epsilon(0)\sigma(x_2)\sigma(x_3)\rangle$ from DMRG for $L=40$. We only show the range of separation in which the lattice operators don't overlap, which is $|x_2|,|x_3|\ge 3$, $|x_2-x_3|\ge 2$. Right: Relative error between our DMRG and CFT. We see that the agreement is better than 10\% in a large region (where all three points are sufficiently distant from each other). This region is shown with a red dashed line in the left plot.}
	\label{fig:eps-sig-sig}
\end{figure}

\section{Estimation of needed sample size}

The DMRG calculation has access to the ground state wavefunction
\beq
\Psi_{\rm GS} = \sum_{n_i=0,1} a_{n_0,\ldots, n_{L-1}} |n_0,\ldots, n_{L-1}\rangle
\eeq
The correlation function observables are computed averaging with respect to this wavefunction.

In the experiment we will not have access to the wavefunction itself but to \emph{collapsed states} $\Psi_1,\Psi_2,\ldots,\Psi_N$. Each $\Psi_i$ is one of the basis states in the $n_i=0,1$ basis and they will appear in the experiment according to the Born rule, i.e. probability to observe a collapsed state $|n_0,\ldots, n_{L-1}\rangle$ equals $|a_{n_0,\ldots, n_{L-1}}|^2$. 

Since all our observables are functions of $n_i$, we can obtain them averaging over a sufficiently large number of collapsed states. But how many collapsed states shall we need to obtain good signal-to-noise ratio? I.e.~how many times shall we have to repeat the experiment? To study this question, we need to study the \emph{variance} of our observables. 

There are two ways to study the variance. The first way is to compute the variance of any observable $\mathcal{O}$ by evaluating the correlation function of its square $\mathcal{O}^2$. It's nice to keep in mind this method as it allows to optimize.

Below we will report the results of a second method, which is more direct and closer to the experiment. As is well known, DMRG algorithms allow to quickly \emph{sample} the wave function, i.e. to generate sequences of random collapsed states distributed according to the Born rule.\footnote{sample() function of the iTensor.jl package.} This is very quick and allows us to simulate the experimental process directly. Statistical errors will go down as $C/\sqrt{N}$, and we can estimate the constant $C$ needed to resolve the signal.

\SR{stopped here}

The result depends on the size of the sample, see Fig.~\ref{fig:sig-sig-sample} for the two point function $\langle\sigma \sigma \rangle$, and Fig.~\ref{fig:eps-sig-sig-sample} for the three point function $\langle\epsilon\sigma\sigma\rangle$.
\begin{figure}[htbp]
	\centering
	\includegraphics[scale=0.5]{fig-sigma-sigma-sample-10e3.pdf}
		\includegraphics[scale=0.5]{fig-sigma-sigma-sample-10e4.pdf}
	\caption{The three-point function $\langle \sigma(0)\sigma(x)\rangle$ from sampling the ground state. Left - sample size $N=10^3$. Right - sample size $N=10^4$. }
	\label{fig:sig-sig-sample}
\end{figure}

\begin{figure}[htbp]
	\centering
	\includegraphics[scale=0.7]{fig-eps-sig-sig-sample-10e4.pdf}
		\includegraphics[scale=0.7]{fig-eps-sig-sig-sample-10e5.pdf}
	\caption{The three-point function $\langle \epsilon(0)\sigma(x_2)\sigma(x_3)\rangle$ from sampling the ground state. Left - sample size $N=10^4$. Right - sample size $N=10^5$. }
	\label{fig:eps-sig-sig-sample}
\end{figure}

\section{Effects of distance variation}
\label{sec:distvar}

To be discussed in the future is the above is judged promising.
 
 \section{Tuning to the critical point}
 \label{sec:tuning}
 
 To be discussed in the future is the above is judged promising.
 


\bibliography{reference.bib}
\bibliographystyle{utphys}

\end{document}
