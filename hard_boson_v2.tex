%%%%%%%
%\documentclass[preprint,aps,nofootinbib,preprintnumbers,amsmath,amssymb,11pt]{revtex4-2}
\documentclass[prd,twocolumn,11pt]{revtex4-1}
%%

%%
\usepackage[colorlinks=true, linkcolor=blue, citecolor=blue, urlcolor=blue]{hyperref}
\usepackage{amsthm}
\usepackage{graphicx}
\usepackage{amsfonts}
\usepackage[figuresright]{rotating}
\usepackage{amssymb}
\usepackage{amsmath}
\usepackage{psfrag}
\usepackage{subfigure}
\usepackage{wrapfig}
\usepackage{multirow}
\usepackage{tabularx}
\usepackage{float}
\usepackage{epsfig}
% \graphicspath{ {images/} }
\usepackage{amsmath}
\usepackage{amssymb}
\usepackage{amsfonts}
\usepackage{graphicx}% Include figure files
\usepackage{dcolumn}% Align table columns on decimal point
\usepackage{bm}% bold math
\usepackage{natbib}
\usepackage{xcolor}
%\textwidth=6.0in 
\newcommand{\half}{\frac{1}{2}}
\newcommand{\beq}{\begin{equation}}
\newcommand{\eeq}{\end{equation}}
\newcommand{\bea}{\begin{eqnarray}}
\newcommand{\ea}{\end{eqnarray}}
\newcommand{\p}{\partial}
\newcommand{\nn}{\nonumber}
\newcommand{\Rm}{\mathbb{R}}
\linespread{1.1}

\def\avg#1{\langle#1\rangle}
\def\Re {\mbox{Re}}
\def\Im {\mbox{Im}}
\def\tr{\mbox{tr}}
\def\nn{\nonumber}
\def\pp{\parallel}
\def\c{\hspace{2pt}}
\def\cN{{{\cal N}}}

\newcommand{\SR}[1]{{\color{red} [SR: #1]}}
\newcommand{\red}{\color{red}}

\begin{document}



\def\thesection{\arabic{section}}
\def\thesubsection{\arabic{section}.\arabic{subsection}}
\numberwithin{equation}{section}

\onecolumngrid
\begin{center}
{\Large {\bf Three-point correlation functions of critical system}}
\vspace{10pt}

{\Large
\bigskip
Junchen Rong$^1$, Slava Rychkov$^2$
}
{\\~
\\${}^1$ CPHT, CNRS, \'Ecole Polytechnique, Institut Polytechnique de Paris, Palaiseau, France\\
${}^2$ IHES, Bures-sur-Yvette }
\end{center}
\section{Introduction}
At the critical point, the two-point function and three-point correlation function of operators look like~\cite{polyakov1969microscopic},
\begin{align}
\langle\sigma(x)\sigma(y)\rangle &= \frac{c_1}{|x-y|^{2\Delta_{\sigma}}}, \quad {\rm with} \quad |x-y|:=\sqrt{\sum_i (x^i-y^i)^2}.\nonumber\\
\langle\sigma(x_1)\sigma(x_2)\epsilon(x_3)\rangle &=\frac{c_2}{|x_1-x_2|^{2\Delta_{\sigma}-\Delta_{\epsilon}}|x_1-x_3|^{\Delta_{\sigma}}|x_2-x_3|^{\Delta_{\sigma}}}.
\end{align}
Here, $c_1$ and $c_2$ are arbitrary constants and $\Delta_{\sigma}$ and $\Delta_{\epsilon}$ are the so-called scaling dimensions, which is related to the critical exponents of the critical system.
The correlation functions above are defined as the connected correlations, such as
\begin{align}
    \langle ABC\rangle= \langle ABC\rangle_0-\langle A\rangle_0\langle BC\rangle_0-\langle B\rangle_0\langle AC\rangle_0-\langle C\rangle_0\langle AB\rangle_0+2 \langle A\rangle_0\langle B\rangle_0 \langle C\rangle_0.
\end{align}

For a quantum spin chain of length $L$ with a periodic boundary condition, the formula get modified to be 
\begin{align}
    \langle\sigma(x_1)\sigma(x_2)\rangle& = c_1\left|\frac{1}{2L\sin(\frac{\pi|x_1-x_2|}{L})}\right|^{2\Delta_{\sigma}}\nonumber\\
\langle\sigma(x_1)\sigma(x_2)\epsilon(x_3)\rangle&=c_2 \frac{1}{\bigg|2L\sin(\frac{\pi|x_1-x_2|}{L})\bigg|^{2\Delta_\sigma-\Delta_\epsilon}\bigg|2L\sin(\frac{\pi|x_2-x_3|}{L})\bigg|^{\Delta_\epsilon}\bigg|2L\sin(\frac{\pi|x_1-x_3|}{L})\bigg|^{\Delta_\epsilon}}
\end{align}
Now the bracket denotes the vacuum expectation values. These are predictions from conformal symmetry; our goal is to verify these formulas. 
\section{Transverse field Ising model and DMRG}
We can check these predictions using the DMRG simulation of the transverse field Ising model. 
The transverse field Ising model has the Hamiltonian
\begin{equation}
    \hat{H}=-J \sum_{i} Z_i Z_{i+1}+h \sum_i X_i.
\end{equation}
We define it on a one-dimensional spin chain, with a periodic boundary condition. When $h>J$, the system is in the disordered phase. When $h<J$, the system is in the spontaneous symmetry-breaking phase. The critical point is at $h=J$, which is described by the two-dimensional Ising CFT.
We use DMRG to calculate the ground state wave function at the critical point, and then measure the correlation function. 
To compare with the lattice result, we need to identify the lattice operators with CFT operators
\begin{align}
    \sigma_i \sim Z_i,\nonumber\\
    \epsilon_i \sim X_i.
\end{align}
We consider a spin chain with length $L=40$. When calculating the three-point function, we set $x_1=1$, $x_2=11$, and vary $x_3$.
The results are summarized in Fig. \ref{correlations}. 
\begin{figure}[htbp]
\centering
\includegraphics[scale=0.3]{ss2ptIsing.png}
\includegraphics[scale=0.3]{sse3ptIsing.png}
\caption{The two-point $\langle\sigma(x_1)\sigma(x_2)\rangle$ and three-point functions $\langle\epsilon(x_1)\sigma(x_2)\sigma(x_3)\rangle$. The dots are measured at the critical point of the transverse field Ising model, and the red curve is the result predicted by the conformal symmetry. We have used $\Delta_\sigma=\frac{1}{8}$ and $\Delta_{\epsilon}=1$.}
\label{correlations}
\end{figure}
\section{Fendley-Sengupta-Sachdev model}
There exists a model that is closer to the Rydberg atom experiments, called the Fendley-Sengupta-Sachdev model. The Hamiltonian is 
\begin{equation}
    \hat{H}=-J \sum_i (d_i^{\dagger}+d_i)+ U \sum_i n_i + V \sum_i n_{i-1}n_{i+1}.
\end{equation}
The Hilbert state on each site is two dimensional, spanned by $|0\rangle$ and $|1\rangle$. The creation operator $d^\dagger$ maps $|0\rangle$ to $|1\rangle$. We have
\begin{equation}
    n_i=d_i^\dagger d_i
\end{equation}
which takes values 0,1 in  $|0\rangle$,$|1\rangle$. We refer to $|0\rangle$ as empty and $|1\rangle$ as occupied site. 

The Fendley-Sengupta-Sachdev model imposes the "hardcore boson constraint" on the Hilbert space, saying that 
\begin{equation}
    n_i n_{i+1}=0.
\end{equation}

Experimentally, the model is realized by a chain of Rydberg atoms. We identify the ground state of the atom as the state not occupied by the boson $|0\rangle$, and the Rydberg excited state as the state occupied by the boson $|1\rangle=d^{\dagger}|0\rangle$. The atoms in the excited state interact with a repulsive $1/r^6$ interactions.  The "hardcore boson constraint" then corresponds to an extremely strong repulsion between excited atoms in neighboring sites; often called Rydberg blockade condition. while $V$ corresponds to a weaker but still not negligible repulsive interaction between excited atoms in next-to-neighboring sites. One can change the coupling $J$ by changing the Rabi frequency of the external laser.
The frequency of the external laser is adjusted so that the detuning away from resonance of the $|0\rangle$ to $|1\rangle$ transition is $U$. (See Sachdev's new book ``Quantum Phases of Matter'').

The phase diagram of this model has been studied using DMRG, see for example~\cite{10.21468/SciPostPhys.6.3.033}.
\begin{figure}[htbp]
\centering
\includegraphics[scale=0.4]{phase_diagram.png}
\caption{The phase diagram of the Fendley-Sengupta-Sachdev model. The plot is reproduced from~\cite{10.21468/SciPostPhys.6.3.033}. }
\label{phasediagram}
\end{figure}

 

\begin{figure}[h!]
	\centering
	\includegraphics[scale=0.3]{ss2ptHB.png}
	\includegraphics[scale=0.3]{ee2ptHB.png}
	\includegraphics[scale=0.35]{sse3ptHB.png}
	\caption{The two-point and three-point functions of the Fendley-Sengupta-Sachdev model.}
	\label{correlationsHB}
\end{figure}

We set up our own study of this model using DMRG. We can identify 
\begin{align}
    d^{\dagger}+d \quad \rightarrow  \quad X.
\end{align}
Here $X$ is the $\sigma^x$ Pauli Matrix. 
In our study, to impose the Rydberg blockade condition, we add a strong repulsive interaction 
\begin{align}
    \hat{H}_{rp}=U_{rp} \sum_i n_i n_{i+1}.
\end{align}

\SR{The text below needs to be either modified to take into account our new understanding of operators}

We can also identify the lattice operators with CFT operators according to,
\begin{align}   
    \sigma_i \sim &\quad  (-1)^i n_i \nonumber\\
    \epsilon_{i+1/2} \sim & \quad (1-n_{i})(1-n_{i+1}).
\end{align}
The $\epsilon$ operator is defined on the bonds. 
We study a lattice with $L=40$ sites. We set the coupling constants to be 
\begin{align}
    J=1,\quad U=0.128, \quad V=-1, \quad {\rm and}\quad U_{rp}=20.
\end{align}
\SR{why is $V$ negative? This does not make sense from the point of view of comparing with Rydberg atoms. Should be redone with a more physical positive $V$. $U_{rp}=20$ seems small compared to the actual $2^6=64$ for the $1/r^6$ potential.}

The two-point $\langle\sigma_0 \sigma_1\rangle$ and three-point functions  $\langle \epsilon_i \sigma_1 \sigma_{11}\rangle$ can be found in Fig.~\ref{correlationsHB}.


\section{A more realistic model}
Finally, we can consider a more realistic model with van der Waals $1/r^6$ interaction between the excited atoms. We will follow the same convention as in~\cite{FangFang:2024eaw},
\begin{equation}
    \hat{H}=\frac{\Omega}{2} \sum_i (d_i^{\dagger}+d_i)- \Delta \sum_i n_i + C_6\sum_{ij}\frac{1}{(d_{ij})^6}  n_{i}n_{j}.
\end{equation}
Here $d_{ij}$ is the distance between the $i$-th and $j$-th sites. Since the sites are arranged to form a circle, we have
\begin{align}
    d_{ij}=a \frac{\sin(\frac{|i-j|}{L}\pi)}{\sin(\frac{1}{L}\pi)},
\end{align}
where $a$ is the lattice constant. We will set $a=1$ for later discussion. 
The sign of $\Omega$ is not important, as it can flipped by redefining the phase of the $|1\rangle$ state,  $|1\rangle \to -|1\rangle$.
%\begin{align}
%d^{\dagger}|0\rangle \rightarrow -d^{\dagger}|0\rangle. 
%\end{align}
The experimentally accessible parameter space is, roughly,
\begin{align}
\Omega/C_6 \in [0,5],\quad \Delta/C_6 \in[-5,5].
\end{align}
The phase diagram near the Ising transition is reported in Fig.~\ref{phasediagramHBlong}.
\begin{figure}[htbp]
\centering
\includegraphics[scale=0.5]{phase_diagram_SSF_LR.png}
\caption{The phase diagram of the Fendley-Sengupta-Sachdev model with $1/r^6$ interaction. \SR{Is it appropriate to call this model FSG model?} \SR{Is there a reason why this diagram is symmetric with respect to $\Delta/C_6=1$ horizontal line?}}
\label{phasediagramHBlong}
\end{figure}
The red circle corresponds to the point studied in~\cite{FangFang:2024eaw}. Where they studied the model at  
$R_b/a=\frac{(C_6/\Omega)^{1/6}}{a}=1.4$ and $\Delta/\Omega=0.97(5)$. 
For future reference, we list here the critical points we studied:
{\\
$(\Omega/C_6,\Delta/C_6)=$\{\{0.04,0.058\},\{0.066666666,0.0785\},\{0.1,0.105\},\{0.1333333334,0.1353333333\},\\
\{0.18,0.186\}, \{0.25,0.26875\},\{0.332,0.4\},\{0.4,0.55\},\{0.45,0.7\},\{0.47,0.8\},\{0.484,0.9\},\{0.488,1.\},\\
\{0.488,1.05\},\{0.474,1.2\},\{0.456,1.3\},\{0.4,1.495\},\{0.3,1.69\},\{0.2,1.83\},\{0.1,1.929\},\{0.04,1.977\}\}\\
}


We can study the correlation function at any point on the critical line. We take
\begin{align}
    \Omega/C_6=0.25,\quad \Delta/C_6=0.26875.
\end{align}

\SR{below needs to be modified to take advantage of our new understanding of symmetry}

We can also modify the map between the lattice operators and CFT operators,
\begin{align}
    \sigma_{i+1/2}\sim &~(n_i-n_{i+1}),\nonumber\\
    \epsilon_{i+1/2} \sim &~n_i+n_{i+1}- (1-n_{i})(1-n_{i+1}).
\end{align}
Due to the symmetry-breaking pattern, we treat two neighbouring sites as a single unit cell. The above operators are defined when $i$ is an even integer.
The results are summarized in Fig.~\ref{correlationsHBlong}, for two point and three point functions.
\begin{figure}[htbp]
\centering
\includegraphics[scale=0.35]{ss_long_range.png}
\includegraphics[scale=0.35]{ee_long_range.png}
\includegraphics[scale=0.35]{ess_long_range.png}
\includegraphics[scale=0.35]{ess_long_range2.png}
\caption{The two-point and three-point functions of the Fendley-Sengupta-Sachdev model with $1/r^6$ interaction.}
\label{correlationsHBlong}
\end{figure}

\appendix

\section{Matching of lattice operators and CFT operators}

Consider any lattice operator $O_{lat}(0)$ on a collection of sites $S$ around $0$. This lattice operator will be matched to a linear combination of CFT operators at point $x=0$:
\beq
O_{lat}(0) = \sum c_\alpha O_\alpha(0)
\eeq
Here $O_\alpha(0)$ are CFT operators and coefficients $c_\alpha$ depend on the lattice operator $O_{lat}$. 
Unit operator will also appear in this list, in case we have a nonzero vev.

Now consider the same lattice operator shifted by $i$ lattice site. The claim is that such a lattice translation is equivalent in the CFT to a translation times the CFT $\mathbb{Z}_2$ to the power $i$. This means that:
\beq
O_{lat}(i) = \sum_{\text{even}} c_\alpha O_\alpha(i) + (-1)^i\sum_{\text{odd}} c_\alpha O_\alpha(i)
\eeq
I could give some rationale behind this rule, but let me skip it for now.

This implies that 
\beq
O_{lat}(i+1)-O_{lat}(i) \sim \sum_{\text{even}} c_\alpha \partial_x O_\alpha(i+1/2) + (-1)^{i+1} \sum_{\rm odd}2 c_\alpha \,O_\alpha(i+1/2),
\eeq
where we expanded differences of CFT operators to the first order in the Taylor expansion. On the other hand
\beq
O_{lat}(i+1)+O_{lat}(i) \sim \sum_{\text{even}} 2 c_\alpha O_\alpha(i+1/2) + (-1)^{i+1} \sum_{\rm odd}c_\alpha \,\partial_x O_\alpha(i+1/2),
\eeq

Hence the operator
\beq
(-1)^{i+1}[n_{i+1}-n_i] \sim \sigma(i+1/2)+ (-1)^{i+1}\partial \epsilon (i+1/2) +\ldots
\eeq
up to unknown proportionality coefficient in the r.h.s. On the other hand
\beq
n_{i+1}+n_{i} \sim 1 + \epsilon(i+1/2)+ (-1)^{i+1} \partial \sigma (i+1/2) +\ldots
\eeq
Since the dimension of $\partial \epsilon$ is sufficiently large compared to the dimension of $\sigma$, the first lattice operator approximates the CFT field sigma well. On the other hand the second operator may have a significant admixture of $\partial \sigma$ which has dimension just a bit higher than $\epsilon$. So it helps to consider further averaging over three sites which kills this admixture:
\beq
n_{i+1}+2n_{i} +n_{i-1}\equiv [n_{i+1}+n_{i}]+ [n_{i}+n_{i-1}] \sim 1 + \epsilon(i)+ 0 \times \partial \sigma (i+1/2) +\ldots
\eeq

The latter operator is somewhat inconvenient to work with because it has range 2. One can construct a good operator of range 1 (only two points), but this requires finetuning of coefficients.

Namely consider three lattice operators $O_1 = n_0+n_1, O_2=n_0-n_1, O_3=n_0 n_1$ and expand them in CFT operators $O^{CFT}_1=\epsilon(1/2), O^{CFT}_2=\sigma(1/2), O^{CFT}_3=\partial\sigma(1/2)$ (we omit any other CFT operator as having higher scaling dimension). We have 
\beq
O_a \sim M_{ab} O^{CFT}_b
\eeq
where the mixing matrix has the form
\beq
M=\begin{pmatrix}
	*& 0 & *\\
	0 & * & 0 \\
	* & * & * 
	\end{pmatrix}
	\eeq
where $*$ denotes a generically nonzero coefficient. This shows that there exists a linear combination of lattice operators
$O_1 + a O_2 + b O_3$ which matches only on $\epsilon(1/2)$, without admixtures of $\sigma$ or $\partial\sigma$.

To find this linear combination one can proceed as follows. First find a linear combination $O_3'$ of $O_3$ and $O_2$ which does not have admixture of $\sigma$. This admixture will show up as strong oscillating behavior $(-1)^i/i^{1/4}$ in the two point function.

Then find a linear combination of $O_1$ and $O_3'$ which does not have an admixture of $\partial\sigma$. This admixture gives oscillating behavior $(-1)^i/i^{2.25}$.

In practice what one should do is to measure all 9 two-point functions $\langle O_a(0) O_b(i)\rangle$ and then start looking at mixings.

\section{Algorithm for determining the best operator for $\epsilon$}

Fix $k$. Consider lattice operators supported on $k$ points $0$,...,$k-1$. There are a total of $2^n$ operators, including the identity. Write operators in the basis of products of $n_i$. Only consider "non-rare" operators which do not involve consecutive products $n_i n_{i+1}$, $n_i n_{i+1} n_{i+2}$ etc since those will be very sensitive to rare fluctuations violating Rydberg blockade.

For example for $k=3$ we have 4 non-rare lattice operators
\beq
1,n_0,n_1,n_2,n_0n_2
\eeq
For each of non-rare operators $L_i$, determine the coefficients in the expansion into the leading CFT operators (putting them at $x=(k-1)/2$ for definiteness)
\beq
L_i = a_i 1+b_i \epsilon+ c_i \sigma +d_i \partial\sigma
\eeq
Then impose that $\sum x_i L_i\sim \epsilon$, which gives
\beq
x_i a_i=x_ic_i=x_i d_i=0, \quad x_i b_i=1
\label{eq:constr}
\eeq
Then compute the variance
\beq
(\sum x_i L_i)^2 = \sum x_i x_j L_i L_j\to \sum x_i x_j A_{ij}
\label{eq:var}
\eeq
where $A_{ij}$ is obtained as the unit operator contribution of $L_iL_j$. It can be obtained by expanding $L_iL_j$, replacing all rare operators by zero, and using $a_i$ coefficients above for the non-rare operators.
Finally minimize the variance \eqref{eq:var} over all $x_i$ satisfying constraints \eqref{eq:constr}.
Note that this variance is the one relevant at large separation.



%\begin{multicols}{2}
%\twocolumngrid
%\bibliographystyle{JHEP}
\bibliography{reference.bib}
%\end{multicols}
\end{document}



